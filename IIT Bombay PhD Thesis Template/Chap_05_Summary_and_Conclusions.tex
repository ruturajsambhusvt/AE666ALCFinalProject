\chapter{Summary and Conclusions}

In this thesis, the utilization of complete polarimetric SAR information for the estimation of snowpack parameters is described. Snow parameters in mountain areas are particularly sensitive to changes in environmental conditions. Timely gathering information about snow parameters and their temporal and spatial variability represents a significant contribution in climatology, local weather, avalanche forecasting and for the hydropower production in high mountainous areas. Conventional and ground-based methods represent only exact location measurements of field observations which may or may not be representative of a large area or basin. Due to the strong spatial and time-dependent dynamics of snow cover, frequent observation cycles are necessary. The sensitivity of microwave scattering to the characteristics of snowpack makes RADAR remote sensing a boon to understand a wide range of environmental issues related to the physical condition in high mountainous areas. Especially, the potential for retrieving snow parameters with a high spatial and/or temporal resolution corresponds to become an important input to snow avalanche forecasting, hydrological and meteorological modeling. 

Synthetic aperture radar (SAR) imaging technology is one of the most important advances in space-borne radar remote sensing during recent decades. In the present investigation, dual-~polarimetric (HH/VV) coherent TerraSAR-X (X-band) and full polarimetric Radarsat-2 (C-band) datasets have been used. Manali- Dhundhi region of Indian Himalaya is considered as a study area for this research work. Field data was collected synchronous with the satellite passes (Appendix-II). Snow parameters such as wetness, density, depth and snow permittivity have been measured using the snow fork instrument over the study area. Detailed analyses of  Microwave interaction with snow covered terrain and different scattering mechanisms are described in ~\cref{sec:2.3}, in order to understand the physical characteristics of snowpack parameters.

In this thesis, four major algorithms were presented for the estimation of snow wetness, snow surface dielectric constant and snow density. The algorithms have been proposed and validated using polarimetric SAR data and near real time in-situ measurements. The methodologies have been clearly explained in ~\cref{sec:3} comprising of four separate sections. The results obtained from these approaches have been meticulously presented with detailed discussions in ~\cref{sec:4} pertaining to the corresponding sections.

\begin{itemize}
	\item A new methodology for the estimation of snow wetness using dual-~polarimetric (HH/VV) coherent high frequency (9.6 GHz) SAR data has been proposed.  
	
	\item A new novel algorithm has been proposed to estimate snow wetness from full polarimetric SAR data. The proposed model was applied to Radarsat-2 fine resolution full-polarimetric data sets acquired over the Indian Himalayan region for three consecutive years.   
	
	\item Snow surface dielectric constant estimation methodology from full-polarimetric SAR data is proposed. The dominant scattering type amplitude ($\alpha_{s1}$) is used to characterize dominant snow.
	
	\item At final, a new methodology for snow density estimation from C-band full--polarimetric SAR data is proposed. The generalized volume parameter is derived from the double unitary transformation of the coherency matrix. 
	
	\item These research works have been assimilated in the HimSAR software toolbox which is under development and expansion for the cryospheric applications using polarimetric SAR data. This toolbox will be helpful for the cryospheric scientific community to utilize, explore and contribute to the further development of this open source toolbox.
	   
\end{itemize}
\section{Contributions}
During this course of research work, four major contributions were made in-terms of new algorithms development for the estimation of snowpack parameters. 
\begin{itemize}
	\item A new snow wetness estimation algorithm is developed for dual.  
	\item A new novel algorithm has been developed to estimate snow wetness using full polarimetric SAR data.  
	\item Snow surface dielectric constant estimation methodology has been developed for full-polarimetric SAR data.   
	\item A new methodology for snow density estimation from C-band full--polarimetric SAR data is developed.  
\end{itemize}
\section{Scope for future research}
\begin{itemize}
	\item The proposed snowpack parameters estimation algorithm can be extended for multi frequency SAR data by considering all possible scattering mechanisms.
	\item Particularly for the estimation of snow density where the snowpack volume scattering has only been considered for the Radarsat-2 C-band data.  
\end{itemize}
  




 

